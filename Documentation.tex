\documentclass{article}
\usepackage[top=2cm, bottom=2cm, left=2cm,right=2cm]{geometry}
\usepackage[utf8]{inputenc}
\usepackage[charter]{mathdesign}
\usepackage[T1]{fontenc}
\usepackage[french]{babel}
\usepackage[svgnames, x11names]{xcolor}
\usepackage{amsmath}
\usepackage{graphicx}
\usepackage{float}
\usepackage{calrsfs}
\graphicspath{ {images/} }
\usepackage{t}
\usepackage[colorlinks=true,urlcolor=blue]{hyperref}
\begin{document}

\titre{Module pour présentation}
\titreDuCours{T.P. 2}
\coteDuCours{PHQ 360}
\auteurA{Gabriel Desharnais}
\auteurB{  }
\auteurC{  }
\date{November 2016}


\entete

    \begin{pythoncode}{Module presentation}
#! /usr/bin/python3
# coding: utf8
''' Ce module modifie les param|è|tres par d|é|faut de MatPlotLib afin de pouvoir produire
des graphiques adapt|é| pour une pr|é|sentation'''
from matplotlib import rcParams
params = {'axes.linewidth': 1.0,
    'axes.labelsize': 30,
    'font.size': 30,
    'legend.borderaxespad': 0.1,
    'legend.borderpad': 0.1,
    'legend.columnspacing': 0.6,
    'legend.fontsize': 25,
    'legend.handlelength': 0.5,
    'legend.handletextpad': 0.25,
    'legend.labelspacing': 0.1,
    'legend.numpoints': 1,
    'lines.linewidth': 2.0,
    'lines.markersize': 10,
    'xtick.labelsize': 30,
    'ytick.labelsize': 30,
    'axes.unicode_minus': False,
    'font.family': 'sans-serif',#Cette famille peut afficher les accents
    'xtick.major.pad': 4,
    'xtick.major.size': 4,
    'xtick.minor.pad': 4,
    'xtick.minor.size': 2,    
    'ytick.major.pad': 4,
    'ytick.major.size': 4,
    'ytick.minor.pad': 4,
    'ytick.minor.size': 2}
rcParams.update(params)
    \end{pythoncode}
    En effectuant une démonstration avec un cas simple:
    \begin{pythoncode}{Exemple d'utilisation du module}
#! /usr/bin/python3
# coding: utf8
'''Ce fichier permet de montrer le fonctionnement du module |«|presentation|»|'''
import matplotlib.pyplot as plt
from presentation import *

x=list(range(42))
y=[(1/42)*X**2  for X in x]

plt.plot(x,y,'.',label=r'$y=a*x^{2}$')

plt.legend(loc=2)
#nommer les axes
plt.xlabel(r"Temps |é|coul|é| $t$ (s)")
plt.ylabel(r'Position $y$ (m)')
#Enregistrer une image png du graphique
plt.savefig('Exemple.png',bbox_inches='tight')
#Afficher le graphique
plt.show()
    \end{pythoncode}
    Ce qui donne le résultat suivant:
    \begin{figure}[ht]
        \centering
        \includegraphics[width=120mm]{Exemple.png}
        \caption{Ça c'est un graphique présentable}
        \label{BoI}
    \end{figure}
    Les documents sont disponibles sur GitHub : \url{https://github.com/Gabriel-Desharnais/UDSgraphiqueTP/}
\end{document}

